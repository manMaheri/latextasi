%-----------------------------------------------------------------------------------------------%
%
% Maret 2019
% Template Latex untuk Tugas Akhir Program Studi Sistem informasi ini
% dikembangkan oleh Inggih Permana (inggihjava@gmail.com)
%
% Template ini dikembangkan dari template yang dibuat oleh Andreas Febrian (Fasilkom UI 2003).
%
% Orang yang cerdas adalah orang yang paling banyak mengingat kematian.
%
%-----------------------------------------------------------------------------------------------%
 
\var{\judul}{AUDIT KEAMANAN INFORMASI SISTEM INFORMASI KASTAMER PADA UNIT \textit{PAYMENT COLLECTION} MENGGUNAKAN STANDAR ISO 27001}
\var{\judulInggris}{\textit{CARA MEMBUAT TUGAS AKHIR PROGRAM STUDI SISTEM INFORMASI MENGGUNAKAN latex}}
\var{\penulis}{VELLA HERMAN}
\var{\email}{fathirpermana@uin-suska.ac.id}
\var{\nohp}{0852XXXXXXXX}
\var{\nim}{11453201711}
\var{\tahun}{2020}
\var{\pengujipertama}{Tengku Khairil Ahsyar, M.Kom.}
\var{\pengujikedua}{Mustakim, ST., M.Kom.}

\var{\pembimbingpertama}{Inggih Permana, ST., M.Kom.}
\var{\pembimbingpertamanip}{198812102015031006}
% Isi prefik pembimbing pertama dengan "NIP" atau "NIK", tanpa tanda kutip
% Silahkan tanya pembimbing anda
\var{\prefiknomorinduksatu}{NIP}

% Jika tidak ada pembimbing 2 silahkan isi dengan -- saja
\var{\pembimbingkedua}{M. Afdal, ST., M.Kom.}
\var{\pembimbingkeduanip}{151XXXXXX}
% Isi prefik pembimbing kedua dengan "NIP" atau "NIK", tanpa tanda kutip
% Silahkan tanya pembimbing anda
\var{\prefiknomorindukdua}{NIK}

\var{\ketuaSidang}{Nurmaini Dalimunthe, M.Kes.}

\var{\tanggalPersetujuan}{27 Desember 2016}

\var{\tanggalSidang}{20 Desember 2016}
\var{\tanggalSidangInggris}{December 20$^{th}$ 2016}

% Tipe diisi dengan "TUGAS AKHIR" atau "PROPOSAL TUGAS AKHIR", tanpa tanda kutip
\var{\tipeta}{PROPOSAL TUGAS AKHIR}

% Jumlah pembimbing, isi dengan kata "SATU" atau "DUA", tanpa tanda kutip
\var{\jumlahpembimbing}{SATU}

% Isi bidang TA dengan "SATU" atau "DUA", tanpa tanda kutip
% SATU untuk TA bidang RSI (6 BAB)
% DUA untuk TA bidang MSI, BSI, DAMING dan yang sejenis (5 BAB)
\var{\bidangta}{DUA}